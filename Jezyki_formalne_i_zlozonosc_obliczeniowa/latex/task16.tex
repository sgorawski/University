\documentclass{article}

\usepackage{polski}
\usepackage[utf8]{inputenc}
\usepackage{nopageno}

\title{JFiZO, 2. lista zdalna}
\author{Sławomir Górawski}
\date{\today}

\begin{document}

\maketitle

\noindent\textbf{Zadanie 16.}
Czy istnieje wyrażenie regularne $\phi$,
oznaczające jakiś niepusty język regularny,
takie że $L_{a\phi} = L_{\phi b}$?
Czy istnieje wyrażenie regularne $\phi$,
oznaczające jakiś niepusty język
regularny, takie że $L_{a^*\phi} = L_{\phi b^*}$?

\paragraph{1.}
Nie. Dowód przeprowadzimy nie wprost.
Załóżmy, że istnieje wyrażenie regularne $\phi$
takie, że $\phi \neq \emptyset$ i $L_{a\phi} = L_{\phi b}$.
Niech x to największa liczba taka,
że każde słowo $w$ z $L_\phi$ zaczyna się od $a^x$.
Weźmy dowolne słowo $w$ z $L_\phi$
które ma na początku $x$ symboli $a$.
Wtedy słowo $wb$ należy do $L_{\phi b}$,
natomiast nie może należeć do $L_{a\phi}$,
gdyż każde słowo z $L_{a\phi}$ ma na początku
co najmniej $x + 1$ symboli $a$.
Zatem $L_{a\phi} \neq L_{\phi b}$, co daje sprzeczność.

\paragraph{2.}
Tak. Takie wyrażenie to $\phi = a^*b^*$, dla którego zachodzi:

\[
    a^*\phi = a^*a^*b^* = a^*b^* = a^*b^*b^* = \phi b^*,
\]
z czego wynika, że $L_{a^*\phi} = L_\phi = L_{\phi b^*}$.

\end{document}
