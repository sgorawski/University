\chapter{Description of applied solutions}

The entire project consists of two parts:
\begin{itemize}
    \item a library,
    which provides a unified interface for connecting to and performing operations on the blockchains,
    \item a web application using that library,
    which provides a convenient interface for the users and a couple of extra features.
\end{itemize}

Such separation makes it easy to potentially change the user interface, e.g. to a command line one,
while being able to preserve a convenient way of interacting with the blockchains.
It also allows for a better control over the project's dependencies,
-- where the library isn't cluttered with requirements of any web framework
-- and makes testing easier.
Core functionality of the library can be fully verified without being restricted by the user interface. Application tests, on the other hand, can have the library calls mocked out,
enabling them to be performed without access to the cryptocurrency testnets,
which makes setting up particular scenarios easier.

\section{Library}

The library is essentially a set of classes with an unified interface for typical wallet management operations, i.e.:
\begin{itemize}
    \item creating a new wallet or retrieving one,
    \item checking balance on a given wallet,
    \item sending money to other wallets of given currency,
\end{itemize}
which shall be explained in more detail below.

\subsection{Creating wallets}

This operation is performed nearly the same way on either blockchain.
First, a private key is either randomly generated, or recovered from a mnemonic phrase.
Next, a public key is retrieved from the private key, and then an address is derived from it.
The public key is not really relevant to the user,
as the address is used in its stead to provide an additional layer of security \cite{bitcoin-addresses}.
Expected format of these pieces of data may differ (e.g. it's hexadecimal in Ethereum),
but the underlying principle remains the same.
Overall, the private key, the address and a mnemonic phrase
allowing to easily recover this wallet later are returned.

A question may arise whether the mnemonic phrase is required, since a private key
already provides control over a wallet?
Essentially, it is not, and the information stored in it is equivalent,
however its format makes it easier for humans to remember,
than a key which is a long, seemingly random sequence of characters.

Theoretically, this whole procedure is independent from the blockchain,
as a wallet is just a combination of keys and can be generated offline.
In principle, that is the case, as it should be in a decentralized network.
However, there is a catch -- in the case of Bitcoin a wallet's address is imported by a node for indexing.
This will be explained in the following subsection.

\subsection{Checking balances}

Retrieving the amount of money available on a given address works a bit differently in the two networks.
The Ethereum accounts are stateful by nature, storing that information directly in themselves
-- all it takes to view it is a single remote procedure call to our node in the network.

In Bitcoin, however, this is not the case -- only transactions are stored in the blockchain,
balances are not. In order to determine how much money does one have at their disposal,
they would have to crawl the chain tracing the flow of money related to their address,
possibly many blocks in, even to the very beginning of Bitcoin.

In practice Bitcoin node implementations provide a solution for this,
indexing transactions related to given addresses,
which makes it possible to retrieve all unspent transactions outputs \cite{utxos} for an address
without crawling the blockchain every time.
Each of these UTXOs (Unspent Transaction Outputs) can be used to make a transfer,
which means that the balance of a wallet is a sum of their amounts.

\newpage

\subsection{Making transfers}

The last operation is rather easy in the case of Ethereum,
where a transaction is constructed by specifying sender, recipient, amount
and the appropriate gas price.
The latter serves (in a rather indirect way) as fee for block miners
and can be inferred from network activity using a built-in procedure of the node.
There is also \textit{nonce}, which is like an account's transcation counter
-- its appropriate value is just a number of our unconfirmed transactions in the network.

When making a transfer to more than one recipient (let us say there are $n$),
$n$ transactions are created, with \textit{nonce} iteratively incremented for each one.
The downside of this simple approach is that the gas price must be provided separately,
so we end up paying $n$ miner fees for all of the transactions.

As was described earlier, Bitcoin deals with transfers, not with accounts,
and this results in a more complex way of creating a transaction.
Some of our unspent transaction outputs must be selected to serve as input to the new transactions,
and they must be spent entirely.
If their total value is too high,
our own address is added as one of the recipients to receive change.
After that step we end up with a transaction object with some inputs
-- worth in total $x$ BTC
-- and recipients (including us) to receive in total $y$ BTC.
The difference between $x$ and $y$ serves as miners fee, like gas price in Ethereum.
The good aspect of this procedure is that it is only paid once,
no matter the number of recipients.

After creating a transaction object of either currency,
it is signed with our wallet's private key and broadcasted to the network,
waiting to be included in the blockchain.

\subsection{Networking}

The actual communication with the node of a given cryptocurrency's network is conducted via HTTP \cite{http},
using JSON \cite{http} format for remote procedure calls.
As it turned out, the JSON-RPCs \cite{json-rpc} used by Bitcoin and Ethereum are standardized,
sharing the same format for requests and responses and differing only in details.
This allowed to encapsulate network communication details of both currencies in a single class.

For each required procedure call, a request is made to a specific node.
The response is then checked for whether it is associated to this request (by an embedded ID).
Lastly, in case of a success response relevant data is extracted from it and returned,
and in case of an error response, the error is handled in an appropriate way.

\section{Web application}

The web application aims to provide a clean web interface for the users.
It listens to HTTP requests and returns server-side rendered HTML \cite{html} templates as responses.
Users communicate with server using HTML forms \cite{html-forms}, and their requests are handled by appropriate functions,
which usually use the library's methods to interact with the blockchains.

Aside from allowing users to perform basic operations mentioned in the Library section,
the application provides some extra functionality for better user experience \cite{ux}.
First of all, the balance history for all wallets is saved in a database,
allowing the user to see how it changed over time plotted on a graph.
Secondly, the application provides insight into actual value of the crypto assets in dollars
by integration with the Coinlayer \cite{coinlayer} service.
The prices are retrieved from their API \cite{api}
and cached in the database, to avoid flooding the service with requests.
Wallet's balance value is also tracked, saved in the database and
presented to the user in form of a graph.
