\chapter{Summary}

In the last chapter we shall consider some possible uses of the project
and its potential for further development.

First and foremost, the application could be used by individual users
who invest in cryptocurrencies for management of their personal wallets.
The unified interface would provide a comprehensive overview
of digital assets across different blockchains.
To get there, the project would have to be deployed on a publicly available domain,
along with a backend of several network nodes for different blockchains.
Proper configuration of such service is likely a non-trivial task,
but it has already been done by similar solutions,
e.g. the ones discussed in chapter \ref{3}.

Some other improvements in this scenario could be:
\begin{itemize}
    \item Support for more cryptocurrencies
    (many of them are based on Bitcoin or Ethereum,
    so it should not require any major changes).
    \item A more personalizable user interface,
    for instance providing an overview of all wallets
    of a given user on one screen
    (this would likely require storing user information in the database).
    \item Exchanging cryptocurrencies
    (likely requiring some legal approval).
\end{itemize}

As can be seen above, adapting the project for common use
would require some work.
There is however another scenario, for which the project could
be suitable in its current form without a need for significant modifications
-- usage by development teams working in blockchain-related projects,
such as trading platforms for instance.

Development and testing of such projects often requires dealing
with cryptocurrency wallets and making real transactions.
A simple, self-hosted tool such as this one could be quite helpful
in the process.
Managing all desired cryptocurrencies through a unified interface,
with no unnecessary clutter (like user accounts) would likely
provide a convenient way of collaboration
between members of a team.

\newpage

Additional features that would make the project better
for such use are for example:
\begin{itemize}
    \item Support for more cryptocurrencies (same as in first scenario).
    \item A more detailed insight into the blockchains,
    e.g. tracing desired transactions.
\end{itemize}

To conclude, I believe that the project shows potential for further evolution,
although development teams might be a better target group for it
than casual users.
Either way, there are many factors to consider
and many features possibly worth adding.
A core foundation that can be built upon is there,
and fulfills the requirements set at the beginning of the thesis.

Working on this project has been a valuable learning experience,
allowing me to familiarize myself with technicalities of the blockchain technology and practise designing functional user interfaces.
The ever-changing world of cryptocurrencies had seemed difficult
to grasp at times, however in the end proved to be
both a remarkable application of technology
and a promising path for further study.
