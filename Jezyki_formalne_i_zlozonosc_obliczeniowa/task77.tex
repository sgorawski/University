\documentclass{article}

\usepackage[utf8]{inputenc}
\usepackage{polski}
\usepackage{amsmath}
\usepackage{amssymb}
\usepackage{amsthm}
\usepackage{nopageno}
\usepackage{fullpage}

\begin{document}

\begin{flushright}
\textit{Sławomir Górawski}
\end{flushright}

\bigskip

\noindent\textbf{Zadanie 77.}
Niech $L \subseteq \Sigma^*$ będzie CFL.
Czy wynika z tego, że $L_{3/4}$ jest CFL?

\bigskip

\noindent
Nie. Rozwiązanie opiera się na intuicji, że wiele języków,
w których długości co najmniej 3 różnych wystąpień symboli są ze sobą związane,
nie jest CFL (np. $\{ a^n b^n a^n : n \in \mathbb{N} \}$).
$L_{3/4}$ rozumiemy jako język taki że dla każdego $w \in L_{3/4}$
możemy znaleźć $v \in \Sigma^*$, dla którego $|v| = |w|/3$ i $wv \in L$.
Niech:
\[
    L = \{ a^n b^m a^m b^n : n, m \in \mathbb{N} \}.
\]
$L$ jest CFL, ponieważ istnieje generująca go CFG:
\begin{align*}
    & S \rightarrow aSb \; | \; T \\
    & T \rightarrow bTa \; | \; \varepsilon.
\end{align*}
Wtedy:
\[
    L_{3/4} \supset \{
        a^n b^m a^{(m + n) / 2} : n, m \in \mathbb{N} \land n \le m
    \}.
\]
(Do $L_{3/4}$ należą też słowa z $b$ na końcu, dla $n > m$,
ale nie są one istotne dla tego rozwiązania.)
Pokażemy, że $L_{3/4}$ nie jest CFL.

\begin{proof}
    Załóżmy nie wprost, że $L_{3/4}$ jest CFL.
    Użyjemy lematu o pompowaniu dla CFL.
    Niech $p$ -- stała z lematu.
    Weźmy $s = a^p b^p a^p$. $s \in L_{3/4}$.
    Niech $s = uvwxy$ tak że $|vx| \ge 1$ i $|vwx| \le p$.
    Mamy dwie możliwości:
    \begin{enumerate}
        \item $vwx$ zawiera się w pierwszych 2/3 $s$.
        Wtedy $vx$ jest postaci $a^i b^j, \; i, j \in \mathbb{N}$.
        Usuńmy $v$ oraz $x$ z $s$, otrzymując
        $uwy = a^{p - i} b^{p - j} a^p$, gdzie $i + j \ge 1$.
        Z lematu $\forall_{k \in \mathbb{N}} \; u v^k w x^k y \in L_{3/4}$,
        więc w szczególności $uwy \in L_{3/4}$.
        Z definicji $L_{3/4}$ $uwy$ musi być postaci
        $a^n b^m a^{(m + n) / 2}, \; n, m \in \mathbb{N}$.
        Możemy to rozpisać:
        \begin{align*}
            (m + n) / 2 & = p && \text{(Długość ostatniego ciągu symboli $b$)} \\
            (p - i + p - j) / 2 & = p && (n = p - i, \; m = p - j) \\
            2p - i - j & = 2p \\
            i + j & = 0,
        \end{align*}
        co daje sprzeczność z założeniem, że $|vx| = i + j \ge 1$.
        
        \item $vwx$ zawiera się w ostatnich 2/3 $s$.
        Wtedy $vx$ jest postaci $b^i a^j, \; i, j \in \mathbb{N}$.
        Niech $uwy = a^p b^{p - i} a^{p - j}$, gdzie $i + j\ge 1$.
        Z lematu $uwy \in L_{3/4}$.
        Teraz mamy do rozpatrzenia:
        \begin{enumerate}
            \item $i > 0$.
            Wtedy w $uwy$ symboli $b$ jest mniej ($p - i$) niż symboli $a$ przed nimi ($p$).
            Jeśli $uwy$ należy do $L_{3/4}$, to musi być postaci
            $a^n b^m a^{(m + n) / 2}$, gdzie $ n \le m$.
            Dla $uwy$ $n = p$ oraz $m = p - i$, co daje sprzeczność z $n \le m$.
            \item $i = 0$. Wtedy $j \ge 1$.
            Jeśli $uwy = a^p b^{p - i} a^{p - j} = a^p b^p a^{p - j}$ należy do $L_{3/4}$,
            to musi być postaci $a^n b^m a^{(m + n) / 2}$,
            z czego wynikają równości:
            \begin{align*}
                (m + n) / 2 & = p - j && \text{(Długość ostatniego ciągu symboli $b$)} \\
                (p + p) / 2 & = p - j && (m = n = p) \\
                j & = 0,
            \end{align*}
            co daje sprzeczność z wynikającym z założenia $j \ge 1$.
        \end{enumerate}
    \end{enumerate}
\end{proof}

\end{document}
